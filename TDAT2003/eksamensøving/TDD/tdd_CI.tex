\documentclass{article}
\title{TDD}
\begin{document}
    \begin{flushleft}

        \section{TDD generelt}
        Spesifikasjon: tester og krav og to sider av samme sak. Tester kan derfor beskrive krav og vi slipper to artefakter for samme sak. 
        God, ren kode. Mye trygghet. \par 
        \bigskip

        \textbf{Motivasjon}\par 
        \begin{itemize}
            \item tester fungerer utmerket som kravdokumentasjon. I agil ånd gir det ikke mening å duplisere kravdokumentasjonen i både kravdok og Tester
            \item en akseptansetest kan beslriver som scenarior i en user story og funger ebåde som funksjonell test og som kravdok. 
            \item En enhetstest fungerer besdre som dokumentasjon av enn klasse enn javadoc(doc).
            \item Viktig med detaljerte navn på testene.
            \item God, ren kode.  
            \item Bruk av DI tydeligjør avhengigheter, og muligjør mocking. Dette er nødvendig for tester som involverer eksterne systemer
            \item testbar kode er ikke nyttig kun for testene sin skyld, fordi den fremtvinger løse koblinger og høy cohesion
            \item Tester gir en viss trygghet for at koden gjør det den skal
            \item Vi får kjørt koden med en gang
        \end{itemize}

        \bigskip
        \textbf{Testing generelt}
        \begin{itemize}
            \item Vi tester programvare for å validere at systemet:
            \item responderer riktig på alle typer input
            \item yter godt
            \item er brukervennlig
            \item kan installeres i riktig miljø  
            \item ikke brekker ved endringer
        \end{itemize}
        
        \bigskip
        \textbf{Tradisjonell vs smidig testing}
        \begin{itemize}
            \item Utviklere og testere jobber sammen
            \item CI gjør versjonsforskjellene små
            \item Dette gjør at det er mindre sannsynelig at det oppstår store feil
            \item Risk poker for å identifisere risiko
            \item Testnivåer
            \item \begin{itemize}
                \item System test: tregt og kostbart, tester fullt integrert system
                \item Integration test: medium: tester større komponenter spiller bra sammen
                \item Unit test: billig og rask: små enheter, ofte klassen, Junit. En del av byggeprosessen CI
                \item og:
                \item Akseptansetest: en type systemtestsom typisk utføres av kunden
                \item Regresjonstest: skal sikre at systemet ikke brekker ved endringer
                \item Smoke-testing: Uformell test, hvor man raskr prøver ut de viktigste delene av et system for å sjekke at alt henger på greip.
                \item Utforskende testing: bruker kunnskap til å utføre kreativ testing
                \item Destruktiv testing: selvforklarende.
                \item Usability testing: teste om et system er lett å bruke. 
                \item Ytelsetesting: handler om å finne ut hvor godt et system yter. Finne responstider osv.
                \item Stabilitetstesting: Finne problemer som oppstr under korte perioder med høyt stressnivå
            \end{itemize}
        \end{itemize}

        \section{Test first}
        \begin{enumerate}
            \item Write a test, watch it fail
            \item Write just enough code to pass the test
            \item Improve the code without changing its behaviour
        \end{enumerate}

        \section{CD}
        \begin{itemize}
            \item Bygge koden automatisk
            \item Etter bygging så kjører alle testene
            \item Gradle, Maven osv
            \item GitLab CI med Docker(hvem som helst kan lage og publisere et Docker image som inkluderer en linux server med all software man trenger.)
            \item yml fil
            \item 
        \end{itemize}

    \end{flushleft}
\end{document}