\documentclass{article}
\title{Autentisering}

\begin{document}
\begin{flushleft}
    \textbf{Single Sign-On}\par
    Logger på en sentral påloggingstjeneste og får en \emph{token} tilbake. 
    Kan bruke samme token om igjen for å slippe å logge inn igjen. \par 

    \textbf{Autentisering i REST-tjenester}\par  
    \begin{itemize}
        \item Sesjonsbasert: enkelt, bruker servertilstand(ikke helt REST-ful), fungerer ikke over mange REST-tjenester(SSO)
        \item Token-basert: kan brukes på mange tjenester(SSO), token må sendes i hvert kall (auth-header), krever sentralt register
        \item Sertifikiat basert(signerte tokens), Rolls Royce, helt tilstandsløs(REST-ful), kan brukes på mange tjenester(SSO) 
    \end{itemize}

    \textbf{Express Middleware}
    \begin{itemize}
        \item /API
        \item Noe usynlig du putter mellom to lag i en app.
        \item I Express kan vi beskytte våre endepunkter med å innstalere en funksjon som kjører for vi får tilgang til endepunktene
        \item Denne funksjonen kan avgjøre om vi skal få tilgang til endepunktene eller sende en feilmelding til klienten
        \item Vi innstalerer en middleware-funksjon ved å bruke \emph{app.use("/thepath", enFunksjon)}.
        \item Da vil enFunskjon kjøre for alle endepunktene som under samme path kan nås.
    \end{itemize}

    \section{Ulike autentiseringer}

    \textbf{Tokenbasert (usignert)}
    \begin{itemize}
        \item Klient kaller REST, men brukernavn og passord
        \item REST generer random Token og lagrer
        \item Klient spør om funksjon med token i header 
        \item Server sjekker token mot lagra og utfører funksjonen 
    \end{itemize}

    \textbf{Signaturbasert autentisert (JWT)}
    \begin{itemize}
        \item Klient kaller REST, men brukernavn og passord
        \item Server lager en JWT med en henmmelighet og returnerer til klienten
        \item Klient sender JWT i headeren til funksjonen til server 
        \item Server sjekker JWT signatur, får brukerinfo fra denne, og responderer
    
    \end{itemize}
    Server slipper å lagre tokenet på serveren. Vi sjekker ikke mot database, vi sjekker bare om signatur er gyldig\par 
    \bigskip 

    \textbf{JWT (JSON Web Token)}\par 
    Når vi lager JWT, lager vi en JSON. Vi fyller ut noen claims som (navn, value)par.
    Disse navnene er reserverte feks. iss(issuer), exp(expiration), sub(subject), aud(audience).
    Så signerer vi dette med et bibliotek, feks jasonwebtoken(node) ekker Java JWT som da generer en header og en signatur
    Vi ender opp med en kodet streng i tre deler, separert med punktum. 


    \section{OAuth2}

    Åpen protokoll som gjør at vi som utvikler applikasjonen kan bruke påloggsfunksjonalititeten til en Identity provider,
    feks, Google, Facebook, Twitter og mer.
    OAuth2 bruker også signerete tokens som blir plassert i headeren. 
    Kan bruke Passport i Node for å gjøre dette. 
    

\end{flushleft}    
\end{document}