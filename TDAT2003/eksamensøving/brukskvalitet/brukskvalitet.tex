\documentclass{article}
\begin{document}

\begin{flushleft}
\textbf{Brukskvalitet for nettsteder}
\bigskip

Arealfordeling på skjermen
\begin{itemize}
    \item Innholdet skal stå i fokus
    \item Man kan ta en \emph{enkelthetstest}: gp igjennom alle elementene i designet, og fjerne dem ett av gangen, hvis designet funker like bra uten dem, fjern dem for alltid. 
    \item En bestemt bruker leser antagelig i F-mønster til vanlig
    \item \textbf{Sidebredde}: det er umulig å forutsi hvilken sidebredde en bruker har. 
\end{itemize}
\textbf{Noen retningslinjer for godt flytende design}\par
Angi elementene i prosent av nettleservindu, ikke bruke absolutte mål. 
Designet må fungere for andre fontstørrelser. 
I CSS can du bruke em, som er høyden på den fonten de bruker.
Tenk på grafikk, og lag eget utskriftsdesign.
\par
\bigskip
\textbf{Tekst på nett}
\par
Skriv kort, skriv venstretungt. Man bør fordele tekst over flere nettsider. 
Start med konklusjonen, pass på at teksten er lett og rask og lese. 
\par
\bigskip
\textbf{Koding og universell utforming}
\par
Bruk HTML elementer slik de var ment til å brukes. 
Om det er mye informasjon som repeteres, ha muligheten til å hoppe over dette.
Sette tom \emph{alt tag} istedenfor å ikke sette en i det hele tatt. Da vet nettleseren at den skal hoppe over.





\end{flushleft}
\end{document}