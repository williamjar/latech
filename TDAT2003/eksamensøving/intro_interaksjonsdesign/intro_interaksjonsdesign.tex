\documentclass{article}

\begin{document}
\begin{flushleft}
\section{\textbf{Introduksjon til interaksjonsdesign}}
\textbf{MMI}\par
Startet på tidlig 80-tall. Interaksjonsdesign er både et eget fagfelt som i stor grad overlapper med MMI, og en prosess
med metoder og teknikker vi kan bruke for nettopp å lage gode brukergrensesnitt. \par
\bigskip
\textbf{Hvorfor er brukergrensesnitt viktig?} \par
Brukergrensesnitt øker sannsynneligheten for lukke i markedet av produktet.
Ansatte kan være mer effektive i arbeidsoppgavene sine og trives bedre på jobb. 
Det gjøre at man kan bruke mindre tid på brukerstøtte. 
Noen ganger kan det være vanskelig å overtale sjefene dine til å bruke tid på interaksjonsdesign, bla. fordi man ikke merker innsparinger i brukerstøtte før etter produktet er levert. 
\emph{Design av brukergrensesnitt er ikke det samme som grafisk design. Interaksjonsdesign handler om å utforme samhandlingen.}\par
\bigskip
\section{\textbf{Interaksjondesign i dybden}}
Ofte forkortet IxD. \emph{Brukeropplevelsen, brukskvalitet og universell utforming.}
Handler om å designe samhandling mellom mennesker og digitale systemer, objekter, tjenester eller miljøer. 
Interaksjondesign er en designvitenskap.\par
Typisk består selve prosessen av følgende trinn \emph{(gjentas)}:
\begin{itemize}
    \item definere krav som skal lages
    \item designe alternativer
    \item lage prototyper
    \item evaluering
\end{itemize}
Dette betyr at interaksjonsdesign er en iterativ prosess, i likhet med andre moderne utviklingsmetoder. 
Interaksjonsdesign har en samling prinsipper. 
\begin{itemize}
    \item Synlighet: vise hvor ting er 
    \item Feedback: systemet signaliserer
    \item Begrensinger: legge inn begrensinger i interaksjonen, slik at det ikke oppstår feil
    \item Konsistens: lage ting slik at brukere gjør lignende ting på lignende måter
    \item Tilbydelser \emph{affordances}: handler om hvilke forventinger man har, man forventer at en knapp kan trykkes på
\end{itemize}
\bigskip
\bigskip
\bigskip
\textbf{1. Brukskvalitet} - \emph{(viktigste delen av interaksjonsdesign)}
\bigskip
\begin{itemize}
    \item lett å lære seg 
    \item effektivt å bruke
    \item lett å huske
    \item fører til få brukerfeil og gjør det lett å hente seg inn om feil oppstår
    \item behagelig å bruke
\end{itemize}
Se Brukskvalitet PDF.\par
\bigskip

\textbf{2. Universell utforming} \par
Se universell utforming PDF.
\bigskip

\textbf{3. Brukeropplevelsen} \par
Brukeropplevelsen, eller UX som det ofte forkortes til, flytter fokuser fra ren brukskvalitet, og over til hvilke følelser systemet gir oss.
Psykologien bak systemet. 
\begin{itemize}
    \item \textbf{totalinntrykk}: liker brukeren systemet?
    \item \textbf{brukskvalitet}: er systemet effektivt å bruke, oppfører det seg som forventet? Kan man bruke det uten å bli frustrert?
    \item \textbf{følelser}: hvordan føler brukeren seg etter å ha bruklt produktet en stund? Glad motivert, underholdt, avslappet, stresset?
    \item \textbf{estetikk}: opplever brukeren systemet estetisk tiltrekkende? 
\end{itemize}

\end{flushleft}
\end{document}