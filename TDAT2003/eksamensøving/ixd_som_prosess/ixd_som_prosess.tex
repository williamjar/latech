\documentclass{article}
\title{IxD som prosess}
\begin{document}
\begin{flushleft}
\textbf{Interaksjonsdesign som prosess}
\begin{itemize}
    \item 1. definere krav til det som skal lages
    \item 2. designe alternativer
    \item 3. lage prototyper
    \item 4. evaluering
\end{itemize}


\textbf{Brukerorientert utvikling}\par
Viktig å innvolvere brukeren når vi definerer(1) og når vi evaluerer(4). 
For å ha fokus på brukerne og deres behov gjennnom hele utviklingen.
Må være forsiktig, er dette representativt for alle brukere? 
Om man tar hensyn til en bruker, fjerner man behovene til en annen bruker? 
En måte å demokratisere en utviklingsprosess på. 
Mange av ideene i det agile manifestet for smidig programmvare utvikling harmonerer veldig godt med interakjsonsdesign og brukerorientert utvikling.
Mange lykkes godt med å integrere interaksjonsdesign med smidige metoder som Scrum. 
Designere på scrum teamet, eller ha alle som designere.

\bigskip

\textbf{1. Definere krav}\par
Oppdragsgiverens ide om produktet og brukergrensesnittet er bakteppet for kravdefinisjonen. 
Vi må da finne ut hvem brukerne av systemet er. Feks. alder, kjønn, kultur, utdanning, fysiske trekk, psykologiske trekk

\bigskip

\textbf{Personaer}\par 
Samling av personer som representerer de mest typiske brukermotivasjonene. Beskrive fysiske og psykiske trekk.
Man beskriver hvem personaen er, hva som
motiverer ham/henne og hvordan personaen i dag går fram for å løse problemene systemet er
tenkt å hjelpe til med.
\bigskip

\textbf{Datainnsamling}\par
Når vi skal finneut informasjon om brukere er det mange måter å gå frem på. Vi kan bruke spørreskjemaer, intervju og fokusgrupper, observasjon.
Vi kan også studere eksisterende systemer.
\bigskip

\textbf{Å beskrive datainnsamling/brukerhandlinger}\par
Vi vil beskrive hvordan brukeren samhandler med systemet(som krav), og hvordan brukeren samhandler med eksisterende systemer(som datainnsamling).
Ulike teknikker for å beskrive slike samhandlinger.
\begin{itemize}
    \item Use case-diagram
    \item Scenarior/user stories
\end{itemize}

\textbf{2. Designe alternativer}\par 
\emph{Konseptuelt design} \par 
Tenke på hvilke type grensesnitt brukeren møter. 
Høyniva beskrivelse av hvordan et system er organisert.
Etter hvert kan man bli mer konkret i designet. 
Til slutt kan man bevege seg over til mer konkrete forslag til design og prototyper.
For å velge hvilket alternativ som er best, snakke med arbeidsgiver, eller gjøre et valg med utgangspunkt i kvalitet. 
\bigskip

\textbf{3. prototyping}\par 
Tidlig i utviklingen av et grensesnitt, har vi \emph{low fidelity prototyper}:
\begin{itemize}
    \item Wireframes: skisser av mulig grensesnitt
    \item Papirprototyper
    \item Storyboarding (tar utgangspunkt i scnearioer eller use case)
    \item Wizard of Oz: lage et raskts grensesnitt uten å bruke tid på å koble det sammen. Noen trekker trådene.
\end{itemize} 

\emph{High fidelity prototyper}\par 
Virkelighetsnære prototyper er krevende å lage. Bør begrenses til senere iterasjoner. 

Man kan enten bruke og kaste prototypen, eller bruke evolusjonære prototyping der man lar prototypen utvikle seg.\par
\bigskip 
\textbf{Oppsummering}\par 
Vi har sett på de tre første fasene i interakjsonsdesignprossessen. 
Vi må først hente inn så mye relevant informasjon som mulig for å lage god krav til systemet. 
I designfasen lager vi et konseptuelt design som vi bruker til å genere mer konkrete design. 
Vi jobber videre med å forbedre den konseptuelle nodellen og konkrete designforslag på bakgrunn av de forbedrede kravene. 
Vi kommer til evalueringen(steg 4).
Vi fortsetter slik til vi har et grensesnitt som er bra nok.


\end{flushleft}
\end{document}