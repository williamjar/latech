\documentclass{article}
\title{Scrum} 
\begin{document}
\begin{flushleft}
    
    \textbf{\large Det agile manifesto og SCRUM}\par 

    \emph{Vi finner bedre måter å uvikle programvare på ved å gjøre det selv og ved å hjelpe andre med det. Gjennom dette arbeidet har vi lært oss og versette dette.}
    Verdiene i det agile manifesto brukes i Scrum:
    \begin{itemize}
        \item Personer og samspill, fremfor prosesser og verktøy
        \item Programvare som virker, fremfor omfattende dokumentasjon
        \item Kundesamarbeid framfor kontraktforhandlinger
        \item Reagere på endringer, fremfor å følge en plan
    \end{itemize}

    \dotfill

    \textbf{What is scrum?}\par 
    \begin{itemize}
        \item an agile process
        \item team-based
        \item a process that controls caos
        \item improve communication
        \item detect and cause removal og anything that gets in the way of developing products scaleable
        \item a way for everyone to feel good
    \end{itemize}

    \textbf{Scrum - three pillars}\par 
    \begin{itemize}
        \item Transparancy - feedback loop
        \item Inspection - kontroll opp mot sprintmål
        \item Adaption - justeringer underveis
    \end{itemize}
    Three ancilliary roles: stakeholder, managers
    Product backlog - sprint backlog - sprint recursion with daily scrum meeting - potentially shippable product 
    
    \bigskip

    \textbf{Three primary roles of SCRUM}\par
    \begin{itemize}
        \item Product owner: ROI, voice of the customer, managing the product backlog, resposnible for profits, adjust priority, decide release date
        \item Scrum master: wants the team to be successfull(not managing the team), ensure that the team is productive and functional, ensure the process is followed, remove barriers
        \item Scrum team: \emph{cross functional and self organizing}, around 7 members, cross functional, selects the spring goal, has the right to do anything, demos work results to product owner
    \end{itemize}
    Scrum master and product owner can not be the same person, as the product owner might want to introduce new features the scrum master knows there is no time for. 
    \bigskip

    \textbf{Three artifacts}\par
    \begin{itemize}
        \item Product backlog: the master list of all functionality, dynamic, (user stories)
        \item Sprint backlog: list of tasks that the scrum team is committing to complete in the current sprint, based on product backlog
        \item Burndown chart: a chart showing remining work in the sprint backlog, updated every day during the scrum meeting, it is succesfull if the springt backlog is zero at the end of the sprint
    \end{itemize}

    \textbf{User stories}\par 
    Estimated in points, how many poits a team can handle in one sprint. 
    As a role i want to achieve something \par
    \bigskip


   

    \textbf{The systems quality attributes (user stories)}\par 
    \begin{itemize}
        \item accuracy
        \item reliability
        \item robustness
    \end{itemize}
    The time between serious errors, availability, accuracy in calculations, behaviours when errors occour

    \begin{itemize}
        \item performance
        \item efficiency
    \end{itemize}
    Number of simulataneous users, resource use, response duration, expenses

    \begin{itemize}
        \item maintenance
    \end{itemize}
    Troubleshoting, changeability, testability, portability
    
    \bigskip

    \textbf{Four ceremonies}\par
    \begin{itemize}
        \item Sprint planning meeting: everyone present, definition of done, sprint goal
        \item Daily scrum meeting: 15 mins, what have you done, what have you planned, blocks?
        \item Sprint review: everyone present, demo, inspect and adapt, the success will be assesed during this meeting against the sprint goal
        \item Sprint retrospective: evaluation of the sprint. if we could redo, we would do these things the same way, and what would we do differently. Concrete ideas. 
    \end{itemize}


    Scrum process
    \dotfill
    \bigskip

    \textbf{\large Scrum start}\par
    \textbf{Product backlog}\par
    First step of the process. For Product Owner to articulate the product vision. 
    \begin{itemize}
        \item functional requirments: what the system is supposed to do
        \item non-functional requirments: how the system is supposed to be (written as user story in the \emph{product backlog})
        \item features
        \item technical tasks
    \end{itemize}


    \textbf{Sprint planning part one}\par
    The Product Owner and Team (with facilitation from the ScrumMaster) review the high-priority items in the Product Backlog that the Product Owner is interested in implementing this Sprint. 
    The Product Owner and Team also review the “Definition of Done”.
    \bigskip

    \textbf{Sprint planning part two}\par
    Sprint Planning Part Two focuses on detailed task planning for how to implement the items that the team decides to take on
    The Team selects the items from the Product Backlog they commit to complete by the end of the Sprint. Starting from the highest priority.
    The team decides how much work they will commit to complete.
    The Sprint Planning Meeting will often last a number of hours – the team is making a serious commitment to complete the work, and this commitment requires careful thought to be successful
    Once the time available is determined, the team starts with the first item on the Product Backlog – in other words, the Product Owner’s highest priority item and working together, breaks it down into individual tasks, which are recorded in a document called the \emph{Sprint backlog}.

    \bigskip

    \textbf{\large Scrum second step}\par
    \textbf{Scrum daily meeting}\par
    Once the Sprint has started, the Team engages in another of the key Scrum practices: The Daily Scrum. This is a short (15 minutes or less) meeting that happens every workday at an appointed time\par
    \bigskip
    \textbf{Updating sprint backlog and Burndown}\par
    Each team member update their estimate of the amount of time remaining to complete their task. 

    \bigskip
    \textbf{Refinement}\par
    One of the lesser known, but valuable, guidelines in Scrum is that five or ten percent of each Sprint must be dedicated by the team to refining (or “grooming”) the Product Backlog. This includes detailed requirements analysis, splitting large items into smaller ones, estimation ofnew items, and re-estimation of existing items. 


    \bigskip

    \textbf{\large Scrum third step}\par
    \textbf{Sprint review}\par
    \emph{Present at this meeting are the Product Owner, Team members, and ScrumMaster, plus
    customers, stakeholders, experts, executives, and anyone else interested. }
    Inspect and adapt activity for the product. Review is an in-depth conversation between the team and Product Owner to learn the situation, to get advice, and so forth. 
    Scrum Master needs to see if the features implemented are good enough for "the definition of done". 
    Do not use too much time on the demo.

    \bigskip

    \textbf{\large Scrum fourth step}\par
    \textbf{Sprint retrospective}\par
    \emph{The Team and ScrumMaster will attend, and the Product Owner is welcome
    but not required to attend.} Involves inspect and adapt regarding the process. It’s an opportunity for the team to discuss what’s working and what’s not working. 
    “What’s Working Well” and “What Could Work Better”

    \bigskip
    \textbf{Updating Release Backlog and Burndown Chart}\par
    At this point, some items have been finished, some have been added, some have new estimates, and some have been dropped from the release goal. 

    \bigskip
    \textbf{Starting the next sprint}\par
    The Product Owner may update the Product Backlog with any new insight. There is no down time between Sprints – teams normally go from a Sprint Retrospective
    one afternoon into the next Sprint Planning the following morning.

    \bigskip
    \textbf{Release sprint}\par
    The perfection vision of Scrum is that the product is potentially shippable at the end of each
Sprint, which implies there is no wrap up work required, such as testing or documentation.
Rather, the implication is that everything is completely finished every Sprint; that you could
actually ship it or deploy it immediately after the Sprint Review. The need for a release sprint is a weakness. 

    \bigskip
    \textbf{Common challenges}\par
    \begin{itemize}
        \item teams extend the sprint
        \item assume that a practice is discouraged just because Scrum does not specifically require it
    \end{itemize}



\end{flushleft}    
\end{document}