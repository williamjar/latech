\documentclass{article}
\title{Scrum} 
\begin{document}
\begin{flushleft}
    
    \textbf{\large Det agile manifesto og SCRUM}\par 

    \emph{Vi finner bedre måter å uvikle programvare på ved å gjøre det selv og ved å hjelpe andre med det. Gjennom dette arbeidet har vi lært oss og versette dette.}
    Verdiene i det agile manifesto brukes i Scrum:
    \begin{itemize}
        \item Personer og samspill, fremfor prosesser og verktøy
        \item Programvare som virker, fremfor omfattende dokumentasjon
        \item Kundesamarbeid framfor kontraktforhandlinger
        \item Reagere på endringer, fremfor å følge en plan
    \end{itemize}

    \dotfill

    \textbf{What is scrum?}\par 
    \begin{itemize}
        \item an agile process
        \item team-based
        \item a process that controls caos
        \item improve communication
        \item detect and cause removal og anything that gets in the way of developing products scaleable
        \item a way for everyone to feel good
    \end{itemize}

    \textbf{Scrum - three pillars}\par 
    \begin{itemize}
        \item Transparancy - feedback loop
        \item Inspection - kontroll opp mot sprintmål
        \item Adaption - justeringer underveis
    \end{itemize}
    Three ancilliary roles: stakeholder, managers
    Product backlog - sprint backlog - sprint recursion with daily scrum meeting - potentially shippable product 
    
    \bigskip

    \textbf{Three primary roles of SCRUM}\par
    \begin{itemize}
        \item Product owner: voice of the customer, managing the product backlog, resposnible for profits, adjust priority, decide release date
        \item Scrum master: ensure that the team is productive and functional, ensure the process is followed, remove barriers
        \item Scrum team: around 7 members, cross functional, selects the spring goal, has the right to do anything, demos work results to product owner
    \end{itemize}

    \textbf{Three artifacts}\par
    \begin{itemize}
        \item Product backlog: the master list of all functionality, dynamic, (user stories)
        \item Sprint backlog: list of tasks that the scrum team is committing to complete in the current sprint, based on product backlog
        \item Burndown chart: a chart showing remining work in the sprint backlog, updated every day during the scrum meeting, it is succesfull if the springt backlog is zero at the end of the sprint
    \end{itemize}

    \textbf{User stories}\par 
    Estimated in points, how many poits a team can handle in one sprint. 
    As a role i want to achieve something \par
    \bigskip
    \textbf{Product backlog}
    \begin{itemize}
        \item functional requirments: what the system is supposed to do
        \item non-functional requirments: how the system is supposed to be (written as user story in the \emph{product backlog})
        \item features
        \item technical tasks
    \end{itemize}



    \textbf{The systems quality attributes (user stories)}\par 

    \begin{itemize}
        \item accuracy
        \item reliability
        \item robustness
    \end{itemize}
    The time between serious errors, availability, accuracy in calculations, behaviours when errors occour

    \begin{itemize}
        \item performance
        \item efficiency
    \end{itemize}
    Number of simulataneous users, resource use, response duration, expenses

    \begin{itemize}
        \item maintenance
    \end{itemize}
    Troubleshoting, changeability, testability, portability
    


    \bigskip


    \textbf{Four ceremonies}\par
    \begin{itemize}
        \item Sprint planning meeting: everyone present, definition of done, sprint goal
        \item Daily scrum meeting: 15 mins, what have you done, what have you planned, blocks?
        \item Sprint review: everyone present, demo, inspect and adapt, the success will be assesed during this meeting against the sprint goal
        \item Sprint retrospective: evaluation of the sprint. if we could redo, we would do these things the same way, and what would we do differently. Concrete ideas. 
    \end{itemize}



    


\end{flushleft}    
\end{document}