\documentclass{article}
\title{IxD som prosess 2}


\begin{document}
\begin{flushleft}

\textbf{\large Evaluere i interaksjonsdesign}\par
Evaluering er siste trinn i interaksjonsdesignprosessen. 
Uansett, er dette noe man kan starte tidligere, feks. allerede når man har den konseptuelle modellen i steg 2
Dette kalles \emph{formativ evaluering}.

\bigskip

\textbf{Hvorfor evaluere}\par
\begin{itemize}
    \item Evaluering avslører feil
    \item Evaluering hjelper degmed å oppdage ting du ikke har tenkt på
    \item Evaluering viser hva brukerne faktisk gjøre, ikke bare det du tror dem kommer til å gjøre
    \item Evaluering hjelper deg å holde fokus på brukerne
\end{itemize}

\bigskip

\textbf{Evaluering vs testing}\par
Tester inngår under evalueringen. Vi bruker evaluering som et overordnet begrep.

\bigskip

\textbf{\large Brukertester: Evaluering som involverer brukere}\par
\bigskip
\textbf{1. Klassiske brukertester}\par
Dette er tester der vi inviterer brukerne til å teste ut en prototype. 
Samtykke og taushetserklæring er viktig.
Få tak i en håndfull personer og be dem løse problemer i systemet. 
\emph{Gjennomføring av en brukertest}\par
Planlegging: hva vil vi teste og hvordan? Hvem er brukerne? Hva skal du måle? Når og hvor?\par
\bigskip

\emph{Testplanen}
\begin{enumerate}
    \item Hva er formålet med brukertesten?
    \item Hvilke funksjonalitet skal testes? 
    \item Hva slags system skal du teste?
    \item Hva slags personer skal være testbrukere? 
    \item Hvor skal testen foregå?
    \item Hva slags testutstyr skal brukeren benytte?
    \item Hvilke oppgaver skal brukerne få? 
    \item Hvilken timeplan skal du følge?
    \item Hvilke spmørsmål skal du stille før og etter testen?
    \item Hvem er med i testteamet? 
    \item Hvordan formidle funnene?
    \item Når skal man møtes for videre aksjon? 
\end{enumerate}

\textbf{Oppgavene}\par 
Helt konkrete oppgaver, delvis åpne oppgaver eller helt åpne oppgaver(fint for å finne reaksjonen til brukeren og hva som blir trykket på).
\bigskip

\textbf{Gjennomføring}\par
I et lite forstyrrende rom, planlegg god tid. 

\textbf{Brukertestingens begrensinger}\par
Brukertester avslører viktige brukskvalitetproblemer, men kan ikke si med sikkerhet hvordan alle brukere eller brukere flest opplever systemet.
Brukertester er i stor grad kvalitative vesener. Det lærer deg om hvordan brukeren opplever og løser oppgaver i systemet ditt.
I mindre grad "bør denne knappen være grå eller blå". For å få svar på dette bør vi ha \emph{kvantitative tester}.

\bigskip

\textbf{2. A/B tester (sammenligningstester)}\par
Dette er en kvantitativ testmetode der man lager to (eller flere) varianter av et design, presenterer
brukere for en tilfeldig variant og rett og slett ser hvilken variant som har best resultat.
Situasjoner som egner seg for A/B-testing er situasjoner der du lett kan vise fram alternative
design for brukere. Merk at vi må ha en konkret måte å måle hvilket design som er best på, ellers er denne metoden
nesten uten hensikt.
\bigskip

\textbf{Andre evalueringsmetoder}\par
\begin{itemize}
    \item Feltstudier i naturlige omgivelser: lar en typisk bruker i felten teste systemet. 
    \item Opportunistiske evaluaeringer: huker tak i noen under utviklingen og spør om en konrekt ide
\end{itemize}
\bigskip
\textbf{\large Evaluering som ikke involverer brukere}\par
\bigskip
\begin{itemize}
    \item Heuristiske gjennomganger: gå igjennom velkjente tommefingerregler og retningslinjer for brukskvalitet.
    \item Prediktive modeller: matematiske lover som forutsier hvor lang tid det tar å flytte musepekeren osv. 
    \item Kodevalidering(følger standarder)
    \item Testing i forskjellige nettlesere og platformer
    \item Testing av fleksibilitet
    \item Testing av grad av universell utforming og brukskvalitet(WAVE)
    \item Testing av feilmeldinger
    \item Skjemavalidering

\end{itemize}
\textbf{Evaluering i praksis}\par 
Etter de første designideeene er på plass spør vi hva brukerne synes.
Lenger ut i prosessen når vi har begynt å få på plass mer av designet, er tiden for klassiske brukertester i kontrolt miljø. 
Da luker vi ut \emph{brukskvalitetsfeil}.
Når produktet begynner å bli mer ferdig, og det meste er på plass, kan det være en god ide å gjennomføre feltstudier. 
Låne ut prototyper og intervju brukerne etter et par uker med produktet ditt.
\bigskip

\textbf{Oppsummering}\par 
Vi har sett på mange former for evaluering. Det er ett bredt bregrep. 
Det er den fjerde fasen i IxD, men man kan gjøre det under hele utviklingen.
Det er lett for en designer å se seg blind på det man lager. 


\end{flushleft}   
\end{document}