\documentclass{article}

\begin{document}
\begin{flushleft}
    

\textbf{Universell utforming}
\begin{itemize}
    \item Lage ting som i utgangspunktet er laget for alle, både de med og uten funskjonshemninger.
    \item Nødvendig for noen, bra for alle
    \item Diskriminierings og tilgjengelighetsloven 9 og 11, stiller krav til universell utforming for både offentlige og private virksomheter. 
    \item Forskrft om universell utforming av IKT løsninger sier at vi må fylle 35 av 61 sukesskriterier i standarden "retningslinjer for tilgjengelig webinnhold" WCAG 2.0.
    \item DIFI håndhever reglene
\end{itemize}

\textbf{Prinsipper ved universell utforming}

\begin{itemize}
    \item 1. Enkel og intuitiv bruk
    \item 2. Forstålig informasjon 
    \item 3. Toleranse for feil
    \item 4. Like muligheter for alle 
    \item 5. Fleksibel i bruk
    \item 6. Lav fysisk anstrengelse
    \item 7. Størrelse og plass for tilgang og bruk 

\end{itemize}

\textbf{I praksis} \par
I praksis er det vanskelig å oppfylle alle disse kravene. 
Det er viktig å huske at det ofte ikke er mer tidkrevende å designe universelt. 
Ikke særlige ekstrakostnader av å ha ha et universelt design.
Ikke bare funskjonshemmede som vil dra nytte av det, det kan feks. være nyttig å ha tekst på videoer uansett.
\par
\textbf{Funksjonsnedsettinger webdesignere bør tenke på}

\begin{itemize}
    \item Synshemninger: all grafikk som formidler informasjon bør ha en tekstekvivalent
    \item Hørselshemninger: møter problemer når de møter lyd som ikke er oversatt til tekst
    \item Kognitive funskjonshemninger: dysleksi, dyskalkuli, ADHD. Bør ha tekst til tale, og ryddige nettsteder.
    \item Beveglsehemmede: må bruke spesielle bord, tastatur el.
\end{itemize}

WebAIM (Web with accesibility in mind) har utvikliket WAVE, et verkøy som forteller hvor tilgjengelig designet av et nettsted er.
\bigskip
W3C har utviklet WCAG. WCAG 1.0 kom i 1999 og WCAG 2.0 kom i 2008. DIFI har utviklet løsningsforslag for WCAG. 


\end{flushleft}

\end{document}