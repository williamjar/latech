\documentclass[11pt]{article}
\title{28, Sources of Magnetic Field}
\begin{document}
    \maketitle
    \begin{flushleft}
        \begin{itemize}
            \item How are magnetic field created? 
            \item We will study permanent magnets and electromagnets(both sources of magnetic fields) in details
            \item 
        \end{itemize}
        A magnetic field exerts force only on a \emph{moving} charge. 
        Similarily, we will see that only \emph{moving} charges \emph{create} magnetic fields. 
        We will begin with a single moving point charge that create a magnetic field. 
        We can use this to determine the 

        \section{Magnetic field of a moving charge}
        A single point charge $q$ is moving with a constant velocity $\vec{v}$. 
        We call the location of the moving charge at a given instant the \textbf{source point} and the point P where we want to find the field, the \textbf{field point}.
        \par \bigskip
        The field point a distance $r$ from a point charge $q$, 
        the magnitude of the \emph{electric} field $\vec{E}$ caused by the charge
        is proportional to the charge magnitude $|q|$ and to $\frac{1}{r^2}$,
        and the direction of $\vec{E}$(for a positive $q$) is along the line from source point to field point.
        The corresponding relationship for the magnetic field $\vec{B}$ of a point charge $q$ moving with a constant velocity has some similarites and some interesting differences.
        \par\bigskip
        Experiments sjpw the magnitude of $\vec{B}$ is also proportional to $|q|$ and to $\frac{1}{r^2}$. 
        \textbf{But} the \emph{direction} of $\vec{B}$ is not along the line from the source point to the field point.
        Instead, $\vec{B}$ is perpendicular to the plane containing this line and the particles velocity vector $\vec{v}$.
        The field magnitude B is also proportional to the particles speed $v$ and the sine of the angle $\theta$. 
        Therefore, the magnetic-field magnitude at the point P is:
        \[B=\frac{\mu_0}{4\pi}\frac{|q|v\sin\theta}{r^2}\]
        Where $\mu_0$ is the magnetic constant. 

        \bigskip

        \textbf{Moving charge: vector magnetic field}\par 
        We can combine both the magnitude and direction of $\vec{B}$ into a single vector by using the vector product.
        To avoid having to say "the direction from the source q to the field point P", we introduce a \emph{unit} vector $\hat{r}$.
        This vector points from the source point to the field point. The unit vector is equal to the $\vec{r}$ from the source to the field point, divided my it's magnitude.
          


    \end{flushleft}
\end{document}