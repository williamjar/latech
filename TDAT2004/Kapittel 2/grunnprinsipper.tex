\documentclass{article}
\title{Intro}

\begin{document}

    \begin{flushleft}
        \section{Grunnprinsipper}

        \subsection{Grunnprinsipper}
        Skal se på grunnprinsipper i forbindelse med datakommunikasjon.

        \subsection{Standardisering }
        I starten ble alt utviklet uten standardisering. Det gikk greit helt til datamaskiner skulle begynne å kommunisere. 
        Standardisering har vært en forutsetning for den raske utviklingen vi har hatt i datakommunikasjon frem til i dag.
        Alt har ikke gått på skinner, det er også en rekke ulemper i forbindelse med standarder. 
        Siden prosesser med å utvikle nye standarder er såpass tidkrevende, kan det hindre ny teknologi. 
        Ikke alltid \emph{den beste løsningingen} som blir valgt som standard, det er fordi det ikke eksisterer en standard som er best i alle sammenhenger.
        Både standardiseringsorganisasjoner og store bedrifter kan sette standarder.

        \subsection{Standardiseringsorganisasjoner}
        \begin{itemize}
            \item ISO
            \item ITU
            \item IETF
            \item E3C
        \end{itemize}
        De skilles på hvem som kan være medlemmer, hvordan de er organisert og hva de har hovedfokus på. 

        \subsubsection{ISO}
        Medlemmene er delt i tre kategorier. Member bodies, består av en standardiseringsorganisasjon fra hvert land. 
        Correspondent members , består av organisasjoner fra hvert land som ikke arbeider med nasjonale standarder.
        Subscriber member, består av land med dårlig økonomi. Disse betaler redusert medlemsavgift. 

        \begin{itemize}
            \item 230 tekniske kommiteer.
            \item 100 000 frivillige
            \item "Frivillige" er ofte plassert der av sine arbeidsgivere for å fremme bedriftens løsninger
            \item Arbeidsform: forslag sirkuleres til alle med fullverdig medlemskap, om det er populært forbereder man et utkast som kan vedtas
            \item Arbeidsområde: Man kan bli ISO sertifisert etter standarder. Man får et ISO nummer. ISO nummer kan også omfatte det andre organisasjoner har skapt.
            \item I Norge distribueres ISO gjennom Standard Norge
            \item ISO står bak OSI modellen.
        \end{itemize}

        \subsubsection{ITU}
        Medlemmene i ITU er i regjeringer som representerer sitt land, instrustri og private organisasjoner. 
        ITU er den gammel organisasjon. 
        \begin{itemize}
            \item Radiokommunikasjon
            \item Standardisering av telekommunkasjon
            \item Utvikling av telekommunikasjon
        \end{itemize}

        ITU arbeider for å påvirke den videre utviklingen av det tradisjonelle telenetter på internasjonalt nivå.
        ITU er den eldste internasjonale organisasjonene vi har. 

        \subsubsection{IEEE}
        IEEE har over 365 000 medlemmer. Blandt disse, en del studentmedlemmer. 
        IEEE har ca 40 Societes and Technical Counsils, som igjen har undergrupper. 
        IEE er ledende innen tekniske områder som utvikling av datamaskiner, biomedisinsk teknologi, telekommunikasjon, elektrisistet, romfart og forbrukerelektrnonikk.
        Satt i gang 900 standarderer som er ute i bruk. Ca. 700 nye er i utvikling. 
        
        \subsubsection{IETF}
        IETF krever ikke medlemsskap. Organisasjonen består av en samling personer med veldig ulik bakgrunn. 
        Nettverksdesignere, operatører, brukere og forskere. 
        IETF arbeider med Internett og nettverk. Alle standarder som utvikles kalles RFC(Request for Commons). 
        IETF har sørget for at utviklingen av Internett har gått raskere enn telekommunikasjonsutvikling.

        \subsubsection{W3C}
        Består av en kombinasjon av medlemsorganisasjoner, heltidsansatte og andre interessegrupper som arbeider sammen for å utvikle webstandarder.
        Har røtter tilbake til CERN, hvor web oppstod. 
        \begin{itemize}
            \item Arkitektur
            \item Interaktivitet
            \item Teknologi og samfunn
            \item Webtilgjengelighet
            \item Kvalitetssikring
        \end{itemize}
        Hver av disse har 4 - 10 undergrupper. Standardene som utvikles har type navn som XML og HTTP.

        \subsubsection{Arbeidsdelingen mellom standardiseringsorganisasjonene}
        

        


        

    \end{flushleft}
\end{document}